\documentclass[a4paper,10pt]{article}
\usepackage[utf8]{inputenc}

\input{../../aux/tex/encabezado.tex}

%opening
\title{Kelly criterion}
\author{Gustavo Landfried}

\begin{document}

\maketitle

\section{}

Definiciones:
\begin{align*}
V_N & := \text{Capital at time } N \\
V_0 & := \text{initial capital} \\
r & := \text{the exponentiation growth (continuous) rate} \\
L &:= \text{number of losses} \\
W &:= \text{number of wins} \\
l & := \text{capital invested (proportion)} \\
x & := \text{odds as } (\text{probability}^{-1})
\end{align*}

Proceso multiplicativo:
\begin{align}
 V_N & = V_0 (1+r)^N = V_0 (1-l)^L (1+xl)^W \\
\end{align}

Tomo logaritmo de ambos lados
\begin{align}
 \log V_0 (1+r)^N = \log V_0 (1-l)^L (1+xl)^W \\
\end{align}

Simplifico
\begin{align}
 N \log (1+r) &= L \log (1-l) + W \log (1+xl) \\
 \log (1+r) &= \frac{L}{N} \log (1-l) + \frac{W}{N} \log (1+xl) \\
\end{align}

En el límite $N \rightarrow \infty$
\begin{align}
 \log (1+r) &= p \log (1-l) + (1-p) \log (1+xl) \\
\end{align}

Sabiendo que esta función es cóncava en $l$, por lo que podemos maximizar el $r$ encontrnado su punto crítico en $l$.
\begin{align}
0 &= \frac{\delta}{\delta l} p \log (1-l) + (1-p) \log (1+xl) \\
\end{align}

Resolviendo (donde $\overset{*}{=}$ vale por la derivada).
\begin{align}
0 &\overset{*}{=} - \frac{x (p+l-1) + p}{(1-l)(xl+1)} \\
0 &=  x (p+l-1) + p \\
0 &=  xp+xl-x + p \\
x - xp - p &=  xl \\
x(1 - p) - p &=  xl \\
\end{align}

El resultado es
\begin{align}
 l = \frac{x(1 - p) - p}{x}
\end{align}




\end{document}
